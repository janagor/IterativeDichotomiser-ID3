\section{Wnioski}

Udało się zaimplementować algorytm ID3.

Wyniki działania algorytmu mogą się w znaczący sposób od zbioru, na którego
podstawie jest tworzone drzewo decyzyjne i są przeprowadzane testy.

Rozmiar zbioru wejściowego nie musi koniecznie oznaczać, że jeden ze zbiorów
jest łatwiejszy do zbadania niż inny. Możliwe jest uzyskiwanie dobrych wyników,
trenując na małym zbiorze, jeśli dobrze są w nim odzwierciedlone właściwości
całej populacji.

Znacząct wpływ na wyniki algorytmu ma to, jak atrybuty obiektów ze zbioru
badanego mogę podzielić zbiór na podzbiory ze względu na klasy tych obiektów.
IM uzyskiwane podzbiory bardziej różnią się od siebie nawzajem, tym lepsze
wyniki możemy uzyskać. Wiąże się to z krokiem działania algorytmu, w którym to
jest wyznaczany najlepszy atrybut pod względem uzyskiwanego zysku zbioru przy
podziale opartym na wartościach tego atrybutu.

W celu sprawdzenia tej hipotezy możnaby wykonać więcej testów na zbiorach, o
których wiemy, jak dobrze różna wartość atrybutów skutkuje różną wartością
klasy obiektu i porównać to ze zbioramy, w których nie ma widocznie sensownych
sposobów podziału zbioru.
