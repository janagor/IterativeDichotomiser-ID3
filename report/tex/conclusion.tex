\section{Wnioski}

Udało się zaimplementować minimax z przycinaniem $\alpha-\beta$.

Głębokość przeszukiwania drzewa znaczącą wpływa na na czas obliczeń.

Z wyników partii nie można jednoznacznie określić, która funkcja heurystyczna
dawała najlepsze rezultaty. Wyniki partii  w większości nie wyglądały, jakby
były skorelowane z głębokością przeszukiwań. Jedynie w sytuacji, gdy głębokość
niebieskiego wynosiła 1 przy badaniu jego działania, wynik różnił się od
pozostałych. Funkcja heurystyczna ,,tight'' dawała najgorsze rezultaty, w
porównaniu do reszty heurystyk dla niebieskiego. Dla białego pozwalała jedynie
trochę częściej uzyskać remis, gdy reszta heurystyk powodowała niemal wyłącznie
porażki.

Przyczyną dziwnych rezultatów badań może być charakter gry do której algorytm
został zastosowany. Gracz niebieski odpowiada na ruchy gracza białego, co być
może jest preferowaną opcją, ponieważ czeka się wtedy tylko na błąd oponenta.
Ustawienie figur na szachownicy wymusza to, że w końcu dojdzie do kontaktu i
serii wymian figur. Czarny mając ostateczny ruch może na tym zyskać. Warto
zaznaczyć, że idealna gra warcab kończy się remisem, więc biały może mieć tutaj
mniejszą korzyść z zaczynania, niż ma to miejsce w przypadku niebieskiego.

W celu lepszego zbadania algorytmu należałoby przeprowadzić testy dla większych
wartości głębokości przeszukiwania drzewa. Dodatkowo możnaby spróbować innych
heurystyk lub zmodyfikować te już użyte.

