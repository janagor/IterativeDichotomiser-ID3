\section{Zadanie}
\subsection{Zadanie}
Zaimplementować klasyfikator ID3 (drzewo decyzyjne). Atrybuty nominalne, testy
tożsamościowe. Podać dokładność i macierz pomyłek na zbiorach: \textbf{Breast cancer} i
\textbf{mushroom}. Dlaczego na jednym zbiorze jest znacznie lepszy wynik niż na drugim?
Do potwierdzenia lub odrzucenia postawionych hipotez konieczne może być
przeprowadzenie dodatkowych eksperymentów ze zmodyfikowanymi zbiorami danych.
Sformułować i spisać wnioski.

\subsection{Wskazówki dotyczące ćwiczenia}
\begin{itemize}
  \item Atrybuty nominalne - każdy atrybut może przyjmować jedną z kilku
    dozwolonych wartości, zakładamy, że wartość atrybutu to napis, np. ``kot'',
    ``a'', ``20-34'', ``>40''.
  \item Testy tożsamościowe - jeżeli atrybut testowany w danym węźle ma np. 3
    dozwolone wartości, np. a, b, c, to z węzła tego wychodzą 3 krawędzie
    oznaczone: a, b, c.
  \item Na tym ćwiczeniu klasyfikator trenuje się na zbiorze trenującym, a
    ocenia jego jakość na zbiorze testującym. Należy losowo podzielić zbiór
    danych na trenujący i testujący w stosunku 3:2.
  \item Jeżeli zbiór danych zawiera numery lub identyfikatory wierszy to należy
    je wyrzucić - nie chcemy uczyć się identyfikatorów wierszy.
  \item Brakujące wartości atrybutów taktujemy jako wartość, np. jeżeli symbol
    ``?'' oznacza brakującą wartość, a symbole ``a'', ``b'' wartości normalne, to z
    naszego punktu widzenia mamy 3 wartości normalne (fachowo: 3 wartości
    atrybutu): ``a'', ``b'', ``?''.
  \item Tak naprawdę to nie musimy rozumieć dziedziny problemu - na wejściu mamy
    napisy, na wyjściu napisy, nie ważne czy klasyfikujemy sekwencje DNA,
    grzyby, czy samochody.
  \item Nazwa pliku ze zbiorem danych jest parametrem algorytmu klasyfikacji,
    kod klasyfikatora powinien być w stanie obsłużyć inny zbiór danych o tym
    samym rozkładzie kolumn (czyli nie należy wpisywać wartości atrybutów ``na
    sztywno'' w kodzie).
  \item W repozytorium ze zbiorami danych zwykle w plikach ``.names'' jest
    napisane, który atrybut to klasa (czyli wartości której kolumny mamy się
    nauczyć przewidywać).
\end{itemize}
