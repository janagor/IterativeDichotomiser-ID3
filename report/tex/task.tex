\section{Zadanie}
\subsection{Zadanie}
Zaimplementować algorytm min-max z przycinaniem alfa-beta. Algorytm ten należy
zastosować do gry w proste warcaby (checekers/draughts). Niech funkcja oceny
planszy zwraca różnicę pomiędzy stanem planszy gracza a stanem przeciwnika. Za
pion przyznajemy 1 punkt, za damkę 10 p. Proszę nie zapomnieć o znacznej
nagrodzie za zwycięstwo.

Przygotowałem dla Państwa kod, który powinien ułatwić wykonanie zadania. Nie
można używać kodu z Internetu, czy bardziej ogólnie, kodu, którego nie jest
się autorem.

Wiem co jest dostępne w Internecie, większość dostępnych implementacji ma cechy
szczególne, po których łatwo je rozpoznać.

Zasady gry (w skrócie: wszyscy ruszają się po 1 polu. Pionki tylko w kierunku
wroga, damki w dowolnym) z następującymi modyfikacjami:

\begin{itemize}
  \item Bicie nie jest wymagane.
  \item Dozwolone jest tylko pojedyncze bicie (bez serii).
\end{itemize}

\subsection{Pytania}

\begin{itemize}
  \item Czy gracz sterowany przez AI zachowuje się rozsądnie z ludzkiego punktu
    widzenia? Jeśli nie to co jest nie tak?
\end{itemize}

Niech komputer gra z komputerem (bez wizualizacji), zmieniamy parametry jednego
z oponentów, badamy jak zmiany te wpłyną na liczbę jego wygranych. Należy
zbadać wpływ:
\begin{itemize}
  \item Głębokości drzewa przeszukiwań
  \item Alternatywnych funkcji oceny stanu (nadal ocena jest różnicą pomiędzy
    oceną stanu gracza a oceną przeciwnika), np.:
    \begin{enumerate}
      \item nagrody jak w wersji podstawowej + nagroda za stopień zwartości
        grupy (jest dobrze jak wszyscy są blisko siebie lub przy krawędzi
        planszy)
      \item za każdy pion na własnej połowie planszy otrzymuje się 5 nagrody,
        na połowie przeciwnika 7, a za każdą damkę 10.
      \item za każdy nasz pion otrzymuje się nagrodę w wysokości: (5 + numer
        wiersza, na którym stoi pion) (im jest bliżej wroga tym lepiej), a
        za każdą damkę dodatkowe 10.
    \end{enumerate}
\end{itemize}
